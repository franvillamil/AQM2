% ----------------------------------------------------
\begin{frame}
  \titlepage
\end{frame}
% ----------------------------------------------------
\note{}

% ----------------------------------------------------
\begin{frame}
\frametitle{Course overview}
\centering

\begin{itemize}[<+->]
\item This is the second part of the quantitative methods sequence
\item Focus on \textbf{applying} statistical tools in practice
\item Less theory, more hands-on work with data
\item Goal: go from research question to answer
\end{itemize}

\end{frame}
% ----------------------------------------------------
\note{}

% ----------------------------------------------------
\begin{frame}
\frametitle{What will you learn?}
\centering

\begin{itemize}[<+->]
\item How to choose the right model for your question
\item How to interpret and visualize model results
\item How to evaluate whether a model is appropriate
\item How to work with different types of data (panel, spatial, etc.)
\item Best practices in computing and reproducibility
\end{itemize}

\end{frame}
% ----------------------------------------------------
\note{}

% ----------------------------------------------------
\begin{frame}
\frametitle{Course structure}
% \centering

\begin{tabular}{ll}
  \textbf{Feb 5} & \asher{Introduction} \\
  \textbf{Feb 12} & \BGyellow<2>{Applied regression} \\
  \textbf{Feb 19} & \BGyellow<3>{Applied regression II (binary)} \\
  \textbf{Feb 26} & \BGyellow<4>{Interpretation and diagnostics} \\
  \textbf{Mar 5} & \BGyellow<5>{Best practices in computing} \only<5>{\textit{\footnotesize (move just before break?)}} \\
  \textbf{Mar 12} & \BGyellow<6>{Panel data I} \\
  \textbf{Mar 19} & \BGyellow<6>{Panel data II} \\
  \textbf{Mar 26} & \BGyellow<7>{Spatial data} \\
  \textit{Easter break} & \\
  \textbf{Apr 9} & \BGyellow<7>{Spatial data} \\
  \textbf{Apr 16} & \BGyellow<8>{Other outcomes} \\
  \textbf{Apr 23} & \BGyellow<9>{Project presentations} \\
  \textbf{Apr 30} & \BGyellow<10>{Exam + Review} \\
\end{tabular}

\end{frame}
% ----------------------------------------------------
\note{}

% ----------------------------------------------------
\begin{frame}
\frametitle{Evaluation}
\centering

\begin{itemize}
\item Problem sets (20\%)
  \begin{itemize}
  \item Started in class, finished at home
  \item Short deadlines
  \end{itemize}
\item Proposal presentation and peer review (10\% + 10\%)
\item Final essay (30\%)
  \begin{itemize}
  \item Small research note (max 3,000 words)
  \item Original data analysis using R
  \end{itemize}
\item Exam (30\%)
\end{itemize}

\end{frame}
% ----------------------------------------------------
\note{}

% ====================================================
\section{The Big Picture}
% ====================================================

% ----------------------------------------------------
\begin{frame}
\frametitle{The research process}
\centering

\vspace{10pt}

\Large
\textbf{Theory} $\longleftrightarrow$ \textbf{Data Generating Process} $\longleftrightarrow$ \textbf{Data}

\vspace{20pt}

\normalsize
\begin{itemize}[<+->]
\item Theories make claims about how the world works
\item These claims imply certain patterns in data
\item We observe data and try to learn about the underlying process
\item Our research strategy connects theory to data
\end{itemize}

\end{frame}
% ----------------------------------------------------
\note{}

% ----------------------------------------------------
\begin{frame}
\frametitle{Theory first, methods second}
\centering

\begin{itemize}[<+->]
\item The research question and theory should drive everything:
  \begin{itemize}
  \item What unit of analysis to use
  \item What variation to look at
  \item What empirical strategy to follow
  \end{itemize}
\item Methods are tools to implement that strategy
\item Common mistake: picking a method and then looking for a question
\item In this course: we learn tools, but always ask \textit{why this tool for this question?}
\end{itemize}

\end{frame}
% ----------------------------------------------------
\note{}

% ----------------------------------------------------
\begin{frame}
\frametitle{What is a Data Generating Process (DGP)?}
\centering

\begin{itemize}[<+->]
\item The rules that govern how data comes to exist
\item Includes:
  \begin{itemize}
  \item The social or political process we study
  \item How observations end up in our dataset
  \end{itemize}
\item We never observe the DGP directly
\item We use statistical models to make inferences about it
\end{itemize}

\end{frame}
% ----------------------------------------------------
\note{}

% ----------------------------------------------------
\begin{frame}
\frametitle{Why do we need statistics?}
\centering

\begin{itemize}[<+->]
\item Our theories deal with processes, not just data
\item Data is a window into the underlying process
\item Statistics helps us:
  \begin{itemize}
  \item Separate signal from noise
  \item Quantify uncertainty
  \item Make valid inferences
  \end{itemize}
\end{itemize}

\end{frame}
% ----------------------------------------------------
\note{}

% ----------------------------------------------------
\begin{frame}
\frametitle{Sources of uncertainty}
\centering

\begin{itemize}[<+->]
\item \textbf{Sampling uncertainty}: We observe a sample, not the population
\item \textbf{Theoretical uncertainty}: Our theories are simplifications
\item \textbf{Fundamental uncertainty}: Some processes are inherently random
\item[]
\item All of these create ``noise'' in our data
\item Statistical models help us deal with this noise
\end{itemize}

\end{frame}
% ----------------------------------------------------
\note{}

% ----------------------------------------------------
\begin{frame}
\frametitle{The logic of statistical inference}
\centering

\vspace{10pt}

\begin{itemize}[<+->]
\item \textbf{Probability theory}: Given a known process, what data will we see?
\item[]
\item \textbf{Statistical inference}: Given observed data, what can we learn about the process?
\item[]
\item We're doing the reverse: from data back to process
\end{itemize}

\end{frame}
% ----------------------------------------------------
\note{}

% ====================================================
\section{Workflow Basics}
% ====================================================

% ----------------------------------------------------
\begin{frame}
\frametitle{Learning to use computers as tools}
\centering

\begin{itemize}[<+->]
\item World of quantitative methods is changing fast
\begin{itemize}
  \item e.g. Claude Code
\end{itemize}
\item I think it'll be more important to be really literate with computers
\item Part of this course will also involve learning how to properly use computers
\begin{itemize}
  \item Not using only RStudio, R Markdown, etc, but being ready to do big data-based projects
\end{itemize}
\item We'll have a session on computing, project management, etc -- but today, some notes on version control
\end{itemize}

\end{frame}
% ----------------------------------------------------

% ----------------------------------------------------
\begin{frame}
\frametitle{The problem: managing files over time}
\centering

\begin{itemize}[<+->]
\item Have you ever had files like this?
  \begin{itemize}
  \item \texttt{thesis\_v1.docx}
  \item \texttt{thesis\_v2\_final.docx}
  \item \texttt{thesis\_v2\_final\_REAL.docx}
  \item \texttt{thesis\_v2\_final\_REAL\_submitted.docx}
  \end{itemize}
\item[]
\item What changed between versions?
\item Which version has the correct analysis?
\item How do you collaborate without overwriting each other's work?
\end{itemize}

\end{frame}
% ----------------------------------------------------
\note{}

% ----------------------------------------------------
\begin{frame}
\frametitle{Version control: a better way}
\centering

\vspace{10pt}

\textbf{Version control} is a system that records changes to files over time

\vspace{15pt}

\begin{itemize}[<+->]
\item One file, complete history
\item Every change is recorded with a description
\item Can go back to any previous state
\item Multiple people can work simultaneously
\end{itemize}

\end{frame}
% ----------------------------------------------------
\note{}

% ----------------------------------------------------
\begin{frame}
\frametitle{Why version control for research?}
\centering

\begin{itemize}[<+->]
\item \textbf{Reproducibility}: Track exactly what you did and when
\item \textbf{Backup}: Your work is safely stored, even if your laptop dies
\item \textbf{Collaboration}: Work with others without email chains of files
\item \textbf{Transparency}: Share your code with the research community
\item[]
\item Many journals now require or encourage sharing code via GitHub
\end{itemize}

\end{frame}
% ----------------------------------------------------
\note{}

% ----------------------------------------------------
\begin{frame}
\frametitle{Git and GitHub}
\centering

\vspace{10pt}

\textbf{Git}
\begin{itemize}
\item A version control system
\item Runs locally on your computer
\item Tracks changes to files
\end{itemize}

\vspace{15pt}

\textbf{GitHub}
\begin{itemize}
\item A web platform that hosts Git repositories
\item Stores your code online
\item Enables sharing and collaboration
\end{itemize}

\end{frame}
% ----------------------------------------------------
\note{}

% ----------------------------------------------------
\begin{frame}
\frametitle{The basic Git workflow}
\centering

\begin{enumerate}[<+->]
\item \textbf{Make changes} to your files (write code, edit text)
\item \textbf{Stage} the changes you want to save
  \begin{itemize}
  \item ``These are the files I want to include in my next snapshot''
  \end{itemize}
\item \textbf{Commit} the staged changes with a message
  \begin{itemize}
  \item A snapshot of your project at this moment
  \end{itemize}
\item \textbf{Push} your commits to GitHub
  \begin{itemize}
  \item Upload your local changes to the cloud
  \end{itemize}
\end{enumerate}

\end{frame}
% ----------------------------------------------------
\note{}

% ----------------------------------------------------
\begin{frame}
\frametitle{Ways to use Git}
\centering

\begin{itemize}[<+->]
\item \textbf{GitHub web interface}: Create repos, upload files, edit directly
  \begin{itemize}
  \item Simple but limited
  \end{itemize}
\item \textbf{Command line}: Most powerful and flexible
  \begin{itemize}
  \item \texttt{git add}, \texttt{git commit}, \texttt{git push}
  \end{itemize}
\item \textbf{RStudio}: Built-in Git integration
  \begin{itemize}
  \item Point-and-click interface
  \end{itemize}
\item[]
\item All do the same thing---choose what works for you
\end{itemize}

\end{frame}
% ----------------------------------------------------
\note{}

% ----------------------------------------------------
\begin{frame}
\frametitle{Assignment 1}
\centering

\vspace{10pt}

\begin{itemize}[<+->]
\item Create a GitHub account (if you don't have one)
\item Create a \textbf{public} repository for this course
\item Set up your README and folder structure
\item Create a simple \texttt{.R} file
\item[]
\item This repository is where you'll submit all your assignments
\item Detailed instructions in the assignment document
\end{itemize}

\end{frame}
% ----------------------------------------------------
\note{}

% ----------------------------------------------------
\begin{frame}
\frametitle{What makes a good analysis?}
\centering

\begin{itemize}[<+->]
\item Clear research question
\item Appropriate data for the question
\item Right statistical model for the data
\item Correct interpretation of results
\item Honest about limitations and uncertainty
\end{itemize}

\end{frame}
% ----------------------------------------------------
\note{}

% ----------------------------------------------------
\begin{frame}
\frametitle{Looking ahead}
\centering

\begin{itemize}[<+->]
\item Next session: Applied regression
\item Regression as conditional expectations
\item Multiple regression and control variables
\item Interaction effects and presenting results
\end{itemize}

\end{frame}
% ----------------------------------------------------
\note{}

% ----------------------------------------------------
\begin{frame}
\frametitle{For next week}
\centering

\begin{itemize}
\item Check readings if needed
\item Review your notes on OLS from AQMSS-I
\item \textbf{Finish Assignment 1}
\end{itemize}

\end{frame}
% ----------------------------------------------------
\note{}

% ----------------------------------------------------
\begin{frame}
\frametitle{}
\centering

Questions?

\end{frame}
% ----------------------------------------------------
\note{}
