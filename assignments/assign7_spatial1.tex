\input{preamble.tex}
\textbf{\large Assignment 7: Spatial Data I}\\\vspace{10pt}
\end{center}

\vspace{10pt}
\noindent
\textbf{\large Instructions:}

\vspace{10pt}
\begin{itemize}
\setlength\itemsep{0pt}
  \item {\color{red}{\textbf{Deadline}}}: \textbf{[DEADLINE - TBD], before class}
  \item Submit your work in a separate folder in your GitHub repository
  \begin{itemize}
    \item You can include only the R file or additional ones (e.g. pdf with results)
  \end{itemize}
  \item \textbf{Always use comments} in your R code -- and use them to answer questions
  \item You are encouraged to work together, but each person must submit their own code
  \item Plan is to start Part 1 in class and complete Part 2 at home
  \item I'll upload a solution file to the website after next class
\end{itemize}

\vspace{20pt}
\tableofcontents
\newpage

% ==========================================================================
\section{Part 1: In-Class (Exploring Spatial Data with \texttt{sf})}
% ==========================================================================

In this lab we work with the \code{world} dataset from the \code{spData} package, which contains country polygons for the entire world along with socioeconomic attributes. This dataset is ideal for a first encounter with \code{sf} objects: it loads directly into R (no file download needed), uses real data, and illustrates the key operations you will use throughout the spatial analysis part of the course.

Load the required packages and data with:
\begin{lstlisting}
library(sf)
library(spData)
data(world)
\end{lstlisting}

Key variables in \code{world}:
\begin{itemize}
  \item \code{name\_long} --- country name (long form)
  \item \code{continent} --- continent
  \item \code{area\_km2} --- country area in km\textsuperscript{2}
  \item \code{pop} --- population estimate
  \item \code{gdpPercap} --- GDP per capita
  \item \code{lifeExp} --- life expectancy
  \item \code{geom} --- geometry column (sf)
\end{itemize}

% --------------------------------------------------------------------------
\subsection{Inspecting an \texttt{sf} object}
% --------------------------------------------------------------------------

\begin{enumerate}[label=\alph*)]
  \item Load the \code{world} dataset and inspect its structure. Run \code{class(world)}, \code{names(world)}, and \code{nrow(world)}. In a comment, describe what makes an \code{sf} object different from a regular R data frame. What is the geometry column, and how is it stored?

  \item Check the coordinate reference system (CRS) with \code{st\_crs(world)}. What EPSG code does the dataset use? In a comment, explain what WGS84 means and why it is the standard CRS for global geographic data.

  \item Use \code{st\_geometry\_type(world)} and \code{unique(st\_geometry\_type(world))} to inspect the geometry type. In a comment, explain what a MULTIPOLYGON is and give two concrete examples of countries that would require multiple polygons to represent their territory.

  \item Produce a quick map of GDP per capita using base R graphics:
  \begin{lstlisting}
pdf("world_gdp_base.pdf")
plot(world["gdpPercap"])
dev.off()
  \end{lstlisting}
  In a comment, describe what you see. Which regions appear wealthiest and which poorest?
\end{enumerate}

% --------------------------------------------------------------------------
\subsection{Attribute operations}
% --------------------------------------------------------------------------

A key feature of \code{sf} objects is that standard \code{dplyr} verbs work on them directly, operating on the attribute table while automatically carrying the geometry column along.

\begin{enumerate}[label=\alph*)]
  \item Using \code{filter()}, create a subset of \code{world} containing only African countries. Call it \code{africa}. How many African countries are in the dataset? Plot \code{africa["gdpPercap"]} using base graphics. In a comment, note whether the country count matches your expectations.

  \item Add a new variable \code{pop\_millions} equal to population divided by 1{,}000{,}000 using \code{mutate()}. Then compute the average GDP per capita by continent using \code{group\_by()} and \code{summarise()}:
  \begin{lstlisting}
library(dplyr)

world = world %>%
  mutate(pop_millions = pop / 1e6)

gdp_by_continent = world %>%
  group_by(continent) %>%
  summarise(mean_gdpPercap = mean(gdpPercap, na.rm = TRUE))

print(gdp_by_continent)
  \end{lstlisting}
  In a comment, note that \code{summarise()} on a grouped \code{sf} object \textit{unions} the geometries by group and keeps the geometry column. To get a plain data frame without geometry, use \code{st\_drop\_geometry()} first.

  \item Sort the African countries by GDP per capita (descending) using \code{arrange()}. Print the top 5 rows with \code{name\_long} and \code{gdpPercap}. Name the five countries in a comment.
\end{enumerate}

% --------------------------------------------------------------------------
\subsection{Simple visualization with \texttt{ggplot2}}
% --------------------------------------------------------------------------

The \code{geom\_sf()} layer in \code{ggplot2} allows you to plot \code{sf} objects using the standard grammar of graphics.

\begin{enumerate}[label=\alph*)]
  \item Make a choropleth map of the world colored by \code{gdpPercap}:
  \begin{lstlisting}
library(ggplot2)

ggplot(world) +
  geom_sf(aes(fill = gdpPercap)) +
  scale_fill_viridis_c(option = "plasma", na.value = "grey80",
                       name = "GDP per capita") +
  theme_void() +
  labs(title = "GDP per capita by country")
ggsave("world_gdp.pdf", width = 10, height = 5)
  \end{lstlisting}
  In a comment, describe the geographic pattern. Which regions appear wealthiest? Which appear poorest?

  \item Make the same map restricted to the \code{africa} object. Use \code{scale\_fill\_viridis\_c()} with \code{option = "magma"} and save as \code{africa\_gdp.pdf}. Describe the variation in GDP per capita across African countries.

  \item Improve the Africa map by adding white country borders: modify \code{geom\_sf()} to include \code{color = "white"} and \code{linewidth = 0.3}. Save as \code{africa\_gdp\_borders.pdf}. In a comment, explain how the border layer improves readability.
\end{enumerate}

\newpage

% ==========================================================================
\section{Part 2: Take-Home (Point Data and Spatial Joins)}
% ==========================================================================

A key task in spatial data analysis is combining point data (events or observations with coordinates) with polygon data (regions or countries). We do this using \textit{spatial joins}: each point is assigned the attributes of the polygon it falls within. In this part we work with geo-coded armed conflict event data.

Download the conflict events data from the course repository:
\begin{itemize}
  \item \href{https://github.com/franvillamil/AQM2/tree/master/datasets/spatial}{\texttt{github.com/franvillamil/AQM2/tree/master/datasets/spatial}}
\end{itemize}

Load it with \code{events = read.csv("conflict\_events.csv")}. Key variables:
\begin{itemize}
  \item \code{event\_id} --- unique event identifier
  \item \code{year} --- year of the event
  \item \code{country} --- country name (character; may not match \code{world} exactly)
  \item \code{longitude}, \code{latitude} --- geographic coordinates (WGS84)
  \item \code{fatalities} --- estimated number of fatalities
  \item \code{event\_type} --- type of event (e.g., battles, violence against civilians)
\end{itemize}

% --------------------------------------------------------------------------
\subsection{Converting tabular data to \texttt{sf}}
% --------------------------------------------------------------------------

\begin{enumerate}[label=\alph*)]
  \item Convert the \code{events} data frame to an \code{sf} object using:
  \begin{lstlisting}
events_sf = st_as_sf(events,
                     coords = c("longitude", "latitude"),
                     crs = 4326)
  \end{lstlisting}
  Run \code{class(events\_sf)} and \code{st\_crs(events\_sf)} to verify it worked. In a comment, explain what \code{st\_as\_sf()} does: what does the \code{coords} argument specify, and what does \code{crs = 4326} mean?

  \item How many events are in the dataset? Use \code{nrow()} and \code{table(events\_sf\$event\_type)} to show the count by event type. In a comment, which event type is most common?

  \item Make a map of conflict events overlaid on the world polygon:
  \begin{lstlisting}
ggplot() +
  geom_sf(data = world, fill = "grey90", color = "white", linewidth = 0.2) +
  geom_sf(data = events_sf, aes(color = event_type),
          size = 0.5, alpha = 0.4) +
  theme_void() +
  labs(title = "Armed conflict events", color = "Event type")
ggsave("conflict_events_map.pdf", width = 10, height = 5)
  \end{lstlisting}
  In a comment, describe the geographic pattern. In which regions are conflict events most concentrated?
\end{enumerate}

% --------------------------------------------------------------------------
\subsection{Spatial join: events to countries}
% --------------------------------------------------------------------------

\begin{enumerate}[label=\alph*)]
  \item Use \code{st\_join()} to assign country attributes to each conflict event:
  \begin{lstlisting}
# Verify both objects share the same CRS before joining
st_crs(events_sf) == st_crs(world)

events_joined = st_join(events_sf, world[, c("name_long", "continent", "gdpPercap")])
  \end{lstlisting}
  Run \code{nrow(events\_joined)} and verify it equals \code{nrow(events\_sf)}. In a comment, explain what \code{st\_join()} is doing: how does it determine which country polygon each event point falls within? Why is checking the CRS before joining important?

  \item Some events may not match any country polygon (e.g., events at sea, on islands, or exactly on a border). Check with \code{sum(is.na(events\_joined\$name\_long))}. What fraction of events has no matching country? In a comment, list two possible reasons why a point might not match any polygon.

  \item Count the number of events per country:
  \begin{lstlisting}
library(dplyr)

events_by_country = events_joined %>%
  filter(!is.na(name_long)) %>%
  group_by(name_long) %>%
  summarise(n_events = n(),
            total_fatalities = sum(fatalities, na.rm = TRUE)) %>%
  arrange(desc(n_events))

print(head(st_drop_geometry(events_by_country), 10))
  \end{lstlisting}
  Report the top 10 countries by number of events. In a comment, are the results consistent with your knowledge of contemporary armed conflicts?
\end{enumerate}

% --------------------------------------------------------------------------
\subsection{Choropleth of conflict intensity}
% --------------------------------------------------------------------------

\begin{enumerate}[label=\alph*)]
  \item Join the event counts back to the world polygon data. Note that \code{events\_by\_country} is still an \code{sf} object, so we first drop its geometry before joining:
  \begin{lstlisting}
library(tidyr)

events_by_country_df = st_drop_geometry(events_by_country)

world_conflict = world %>%
  left_join(events_by_country_df, by = "name_long") %>%
  mutate(n_events = replace_na(n_events, 0),
         total_fatalities = replace_na(total_fatalities, 0))
  \end{lstlisting}
  Verify that \code{nrow(world\_conflict) == nrow(world)}.

  \item Make a choropleth map of conflict event counts by country:
  \begin{lstlisting}
ggplot(world_conflict) +
  geom_sf(aes(fill = n_events), color = "white", linewidth = 0.2) +
  scale_fill_distiller(palette = "Reds", direction = 1,
                       name = "N events", na.value = "grey80") +
  theme_void() +
  labs(title = "Armed conflict events by country")
ggsave("conflict_by_country.pdf", width = 10, height = 5)
  \end{lstlisting}
  In a comment, describe the map. Does the geographic pattern match the event-level map from question 2.1c?

  \item Make a second map using log-transformed counts: use \code{log1p(n\_events)} as the fill variable (so countries with zero events are handled). Use \code{scale\_fill\_distiller()} with \code{palette = "YlOrRd"}, \code{direction = 1}, and \code{name = "Log(events+1)"}. Save as \code{conflict\_log\_map.pdf}. In a comment, explain why the log transformation is useful and what it reveals that the raw count map did not.
\end{enumerate}

% --------------------------------------------------------------------------
\subsection{Discussion}
% --------------------------------------------------------------------------

Answer the following questions as comments in your R script. Each answer should be 2--3 sentences.

\begin{enumerate}[label=\alph*)]
  \item Discuss one limitation of the spatial join approach used in this assignment. For example: what happens to events that fall exactly on the border between two countries? How might you handle events that fall just outside a polygon due to small coordinate imprecisions?

  \item What is the difference between \code{st\_join()} and \code{left\_join()}? What information does each use to match rows, and when would you prefer one over the other?
\end{enumerate}

\vspace{15pt}

% ==========================================================================
\section{Data Sources}
% ==========================================================================

\begin{itemize}
  \item World polygons: \code{world} dataset in the \code{spData} R package\\(\code{library(spData); data(world)})
  \item Conflict events: \href{https://github.com/franvillamil/AQM2/tree/master/datasets/spatial}{\texttt{github.com/franvillamil/AQM2/tree/master/datasets/spatial}}
\end{itemize}

\vspace{15pt}

% ==========================================================================
\section{Submission}
% ==========================================================================

Commit your file to your GitHub repository before the deadline. Put it in a separate folder, e.g.\ \texttt{assignment7}. Make sure your repository is public so I can access it.

Your R script should:
\begin{itemize}
  \item Be well-organized with clear section headers (using comments)
  \item Include all code needed to reproduce your analysis
  \item Include your answers and interpretations as comments
  \item Save any plots to files (using \code{pdf()}/\code{dev.off()} or \code{ggsave()})
  \item Run without errors from top to bottom
\end{itemize}

\end{document}
