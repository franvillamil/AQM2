\input{preamble.tex}

\begin{document}

\thispagestyle{empty}

\begin{center}
\textbf{\Large Problem Set 1: Setting Up Git and GitHub}\\\vspace{10pt}
{\Large Applied Quantitative Methods for the Social Sciences II}\\\vspace{10pt}
Carlos III--Juan March Institute, Spring 2026
\end{center}

\vspace{10pt}
\noindent
\textbf{\large Instructions:}

\vspace{10pt}
\begin{itemize}
\setlength\itemsep{0pt}
  \item {\color{red}{\textbf{Deadline}}}: \textbf{February 12, before class}
  \item This problem set walks you through setting up Git and GitHub
  \item You will use this setup to submit all assignments throughout the course
  \item Complete all the tasks below and send me the link to your GitHub repository
\end{itemize}

\vspace{20pt}

\section{Task 1: Create a GitHub Account}

If you don't already have a GitHub account, create one at \url{https://github.com}.

\begin{enumerate}
  \item Go to \url{https://github.com} and click ``Sign up''
  \item Choose a \textbf{professional username}---you may use this for years in your academic career
  \item Use your university email or a professional email address
  \item Complete the verification process
  \item Optionally: Add a profile picture and brief bio
\end{enumerate}

\vspace{10pt}

\section{Task 2: Create a Repository for This Course}

Create a new \textbf{public} repository. Name it something like \code{aqmss2} or \code{quant-methods-2026}.

\subsection{Option A: Using the GitHub Web Interface}

\begin{enumerate}
  \item Click the ``+'' icon in the top-right corner of GitHub
  \item Select ``New repository''
  \item Fill in the form:
    \begin{itemize}
      \item \textbf{Repository name}: \code{aqmss2} (or similar)
      \item \textbf{Description}: ``Problem sets for AQMSS II, Spring 2026''
      \item \textbf{Visibility}: Select \textbf{Public}
      \item Check the box ``Add a README file''
    \end{itemize}
  \item Click ``Create repository''
\end{enumerate}

\subsection{Option B: Using the Command Line}

First, make sure Git is installed on your computer. Then run:

\begin{lstlisting}
# Create a new directory and initialize Git
mkdir aqmss2
cd aqmss2
git init

# Create a README file
echo "# AQMSS II - Problem Sets" > README.md

# Stage and commit the file
git add README.md
git commit -m "Initial commit"

# Set up the remote repository (create it on GitHub first, without README)
git branch -M main
git remote add origin https://github.com/YOUR-USERNAME/aqmss2.git
git push -u origin main
\end{lstlisting}

\textbf{Note}: Replace \code{YOUR-USERNAME} with your actual GitHub username.

\subsection{Option C: Using RStudio}

\begin{enumerate}
  \item First, create an empty repository on GitHub (without README)
  \item In RStudio: File $\rightarrow$ New Project $\rightarrow$ Version Control $\rightarrow$ Git
  \item Paste your repository URL: \code{https://github.com/YOUR-USERNAME/aqmss2.git}
  \item Choose a location on your computer
  \item Click ``Create Project''
\end{enumerate}

\vspace{10pt}

\section{Task 3: Edit the README File}

Your README is the ``front page'' of your repository. Edit it to include:

\begin{itemize}
  \item Your name
  \item A brief description (e.g., ``Problem sets for AQMSS II, Spring 2026'')
  \item Optionally, a list of what will be in the repository
\end{itemize}

\subsection{Option A: On the Web}

\begin{enumerate}
  \item Click on \code{README.md} in your repository
  \item Click the pencil icon (edit) in the top-right of the file view
  \item Make your changes using Markdown syntax
  \item Scroll down and click ``Commit changes''
  \item Add a commit message like ``Update README with my info''
\end{enumerate}

\subsection{Option B: Command Line}

\begin{lstlisting}
# Edit README.md with any text editor, then:
git add README.md
git commit -m "Update README with my info"
git push
\end{lstlisting}

\subsection{Option C: RStudio}

\begin{enumerate}
  \item Edit the \code{README.md} file in the Files pane
  \item Go to the Git pane (usually top-right)
  \item Check the box next to \code{README.md} to stage it
  \item Click ``Commit''
  \item Write a commit message and click ``Commit''
  \item Click ``Push'' to upload to GitHub
\end{enumerate}

\vspace{10pt}

\section{Task 4: Create a Folder and R File}

Create a folder called \code{problem\_sets} and add your first R file.

\subsection{Option A: On the Web}

\begin{enumerate}
  \item Click ``Add file'' $\rightarrow$ ``Create new file''
  \item In the filename box, type: \code{problem\_sets/ps1.R}
    \begin{itemize}
      \item This creates the folder and file at once
    \end{itemize}
  \item Add the following content:
\begin{lstlisting}
# Problem Set 1
# AQMSS II, Spring 2026
# [Your Name]

# This file will contain my solutions for PS1
print("Hello, Git!")
\end{lstlisting}
  \item Scroll down and commit with message ``Add ps1.R''
\end{enumerate}

\subsection{Option B: Command Line}

\begin{lstlisting}
# Create the folder
mkdir problem_sets

# Create the R file (use any text editor)
# Then stage, commit, and push:
git add problem_sets/ps1.R
git commit -m "Add ps1.R"
git push
\end{lstlisting}

\subsection{Option C: RStudio}

\begin{enumerate}
  \item Create a new folder \code{problem\_sets} in the Files pane
  \item Create a new R script: File $\rightarrow$ New File $\rightarrow$ R Script
  \item Add the header content and save as \code{problem\_sets/ps1.R}
  \item In the Git pane, stage the new file, commit, and push
\end{enumerate}

\vspace{10pt}

\section{Task 5: View Your Commit History}

Check that your commits were recorded properly.

\subsection{Option A: On the Web}

\begin{enumerate}
  \item Go to your repository page on GitHub
  \item Click on ``Commits'' (or the clock icon with a number)
  \item You should see your commits listed with messages and timestamps
\end{enumerate}

\subsection{Option B: Command Line}

\begin{lstlisting}
git log --oneline
\end{lstlisting}

This shows a compact list of your commits.

\subsection{Option C: RStudio}

\begin{enumerate}
  \item In the Git pane, click ``History'' (clock icon)
  \item Browse through your commits
\end{enumerate}

\vspace{10pt}

\textbf{Take a screenshot} of your commit history showing at least 2--3 commits.

\vspace{15pt}

\section{Submission}

Send me an email with:

\begin{enumerate}
  \item The URL of your GitHub repository\\
        (e.g., \code{https://github.com/username/aqmss2})
  \item A screenshot of your commit history
\end{enumerate}

I will check that:
\begin{itemize}
  \item Your repository is \textbf{public}
  \item It contains a README with your name
  \item It has a \code{problem\_sets} folder with at least one \code{.R} file
  \item There are multiple commits in the history
\end{itemize}

\vspace{15pt}

\section{Optional: Installing Git Locally}

If you want to use Git from the command line, you need to install it first.

\subsection{Installation}

\begin{itemize}
  \item \textbf{Mac}: Git comes pre-installed. Or install via Homebrew: \code{brew install git}
  \item \textbf{Windows}: Download from \url{https://git-scm.com/download/win}
  \item \textbf{Linux}: Use your package manager, e.g., \code{sudo apt install git}
\end{itemize}

\subsection{First-Time Configuration}

After installing, configure your identity (run once):

\begin{lstlisting}
git config --global user.name "Your Name"
git config --global user.email "your@email.com"
\end{lstlisting}

\subsection{Cloning an Existing Repository}

If you created the repository on GitHub first, download it to your computer:

\begin{lstlisting}
git clone https://github.com/YOUR-USERNAME/aqmss2.git
cd aqmss2
\end{lstlisting}

\subsection{Configuring RStudio}

To use Git in RStudio:
\begin{enumerate}
  \item Go to Tools $\rightarrow$ Global Options $\rightarrow$ Git/SVN
  \item Make sure ``Git executable'' points to your Git installation
  \item Restart RStudio
\end{enumerate}

\vspace{15pt}

\section{Quick Reference: Common Git Commands}

\begin{tabular}{ll}
\textbf{Command} & \textbf{What it does} \\
\hline
\code{git status} & Show which files have changed \\
\code{git add <file>} & Stage a file for commit \\
\code{git add .} & Stage all changed files \\
\code{git commit -m "msg"} & Save staged changes with a message \\
\code{git push} & Upload commits to GitHub \\
\code{git pull} & Download changes from GitHub \\
\code{git log --oneline} & Show commit history (compact) \\
\code{git diff} & Show changes not yet staged \\
\end{tabular}

\vspace{15pt}

\section{Resources}

\begin{itemize}
  \item GitHub's official guides: \url{https://docs.github.com/en/get-started}
  \item Happy Git with R: \url{https://happygitwithr.com}
  \item Git cheat sheet: \url{https://education.github.com/git-cheat-sheet-education.pdf}
  \item Interactive Git exercises: \url{https://gitexercises.fracz.com}
  \item Pro Git book (free): \url{https://git-scm.com/book/en/v2}
\end{itemize}

\end{document}
