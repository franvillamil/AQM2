\input{preamble.tex}
\textbf{\large Assignment 1: Applied Regression II (Binary)}\\\vspace{10pt}
\end{center}

\vspace{10pt}
\noindent
\textbf{\large Instructions:}

\vspace{10pt}
\begin{itemize}
\setlength\itemsep{0pt}
  \item {\color{red}{\textbf{Deadline}}}: \textbf{February 26, before class}
  \item Submit your work as a \code{.R} file called \code{ps3.R} in your GitHub repository
  \item Use comments in your R code to answer conceptual questions and explain your analysis
  \item For model interpretation, use the \code{marginaleffects} package
  \item You are encouraged to work together, but each person must submit their own code
\end{itemize}

\vspace{20pt}
\tableofcontents
\newpage

\section{Conceptual Questions}

Answer these questions using comments in your R script.

\subsection{Question 1: The Linear Probability Model}

\begin{enumerate}[label=\alph*)]
    \item What is the linear probability model (LPM)? How does the interpretation of the coefficient $\beta_1$ differ from standard OLS with a continuous outcome?
    \item Describe two problems with the LPM. For each, explain why it is a problem and when it matters most.
    \item Under what circumstances might the LPM be ``good enough'' despite its limitations?
\end{enumerate}

\subsection{Question 2: Logistic Regression}

\begin{enumerate}[label=\alph*)]
    \item Write down the logit model. What is being modeled as a linear function of $X$?
    \item Explain in your own words why we cannot interpret logit coefficients as changes in probability.
    \item If a logit model estimates $\beta_1 = 0.8$, what is the odds ratio? Interpret it in words.
\end{enumerate}

\subsection{Question 3: Marginal Effects}

\begin{enumerate}[label=\alph*)]
    \item What is an average marginal effect (AME)? How does it differ from a marginal effect at representative values?
    \item Why are AMEs often similar to LPM coefficients? When might they diverge?
    \item A researcher reports only the log-odds coefficients from a logit model. Why is this insufficient for understanding the substantive findings?
\end{enumerate}

\vspace{15pt}

\section{Applied Analysis: Voter Turnout}

For this assignment, you will analyze voter turnout using data from the American National Election Study (ANES) or a similar dataset on political participation.

You can use the ANES data available through the \code{poliscidata} package, or download it from: \url{https://electionstudies.org/data-center/}

\begin{lstlisting}
install.packages("poliscidata")
library(poliscidata)
data(nes)
\end{lstlisting}

Key variables for analysis (variable names may vary depending on data source):

\begin{itemize}
    \item \code{voted}: Whether the respondent voted (binary: 0/1)
    \item \code{age}: Age in years
    \item \code{educ}: Education level (years or categories)
    \item \code{income}: Household income (categories or continuous)
    \item \code{pid}: Party identification strength
    \item \code{female}: Gender indicator
\end{itemize}

\subsection{Question 4: Data Exploration}

\begin{enumerate}[label=\alph*)]
    \item Load and prepare the data. Create a binary turnout variable if needed. Report the overall turnout rate and sample size.
    \item Create a bar chart showing turnout rates by education level. Comment on the pattern.
    \item Create a table showing turnout rates by age group (e.g., 18--29, 30--44, 45--64, 65+). Discuss the pattern in a comment.
\end{enumerate}

\subsection{Question 5: Linear Probability Model}

\begin{enumerate}[label=\alph*)]
    \item Estimate an LPM with turnout as the outcome and age, education, income, and gender as predictors. Print a summary of the results.
    \item Interpret the coefficient on education. What does it mean in terms of probability?
    \item Check how many observations have predicted probabilities outside $[0, 1]$. Report the minimum and maximum predicted values.
\end{enumerate}

\subsection{Question 6: Logistic Regression}

\begin{enumerate}[label=\alph*)]
    \item Estimate a logit model with the same predictors as the LPM. Print the results.
    \item Report the odds ratios for all predictors using \code{exp(coef(model))}. Interpret the odds ratio for education.
    \item Calculate the average marginal effects using \code{marginaleffects::avg\_slopes()}. Compare these to the LPM coefficients---how similar are they?
\end{enumerate}

\subsection{Question 7: Predicted Probabilities and Visualization}

\begin{enumerate}[label=\alph*)]
    \item Using the logit model, calculate the predicted probability of voting for:
    \begin{itemize}
        \item A 25-year-old woman with low education and low income
        \item A 55-year-old man with high education and high income
    \end{itemize}
    Report both point estimates and 95\% confidence intervals.
    \item Use \code{marginaleffects::plot\_predictions()} to plot the predicted probability of voting across values of education, holding other variables at their means. Save the plot.
    \item Create a similar plot showing predicted probabilities across age for different education levels (hint: use \code{condition = c("age", "educ")}). Save the plot and discuss the patterns in a comment.
\end{enumerate}

\subsection{Question 8: Model Comparison}

\begin{enumerate}[label=\alph*)]
    \item Using \code{modelsummary()}, create a table comparing the LPM and logit models side by side.
    \item Briefly discuss in a comment: For this dataset, do the LPM and logit models lead to different substantive conclusions? When might the differences matter?
\end{enumerate}

\vspace{15pt}

\section{Submission}

Commit your \code{ps3.R} file to your GitHub repository before the deadline. Make sure your repository is public so I can access it.

Your R script should:
\begin{itemize}
  \item Be well-organized with clear section headers (using comments)
  \item Include all code needed to reproduce your analysis
  \item Include your answers to conceptual questions as comments
  \item Save any plots to files (e.g., using \code{ggsave()})
  \item Run without errors from top to bottom
\end{itemize}

\end{document}
